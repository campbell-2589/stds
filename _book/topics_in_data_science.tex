% Options for packages loaded elsewhere
\PassOptionsToPackage{unicode}{hyperref}
\PassOptionsToPackage{hyphens}{url}
%
\documentclass[
]{book}
\usepackage{lmodern}
\usepackage{amssymb,amsmath}
\usepackage{ifxetex,ifluatex}
\ifnum 0\ifxetex 1\fi\ifluatex 1\fi=0 % if pdftex
  \usepackage[T1]{fontenc}
  \usepackage[utf8]{inputenc}
  \usepackage{textcomp} % provide euro and other symbols
\else % if luatex or xetex
  \usepackage{unicode-math}
  \defaultfontfeatures{Scale=MatchLowercase}
  \defaultfontfeatures[\rmfamily]{Ligatures=TeX,Scale=1}
\fi
% Use upquote if available, for straight quotes in verbatim environments
\IfFileExists{upquote.sty}{\usepackage{upquote}}{}
\IfFileExists{microtype.sty}{% use microtype if available
  \usepackage[]{microtype}
  \UseMicrotypeSet[protrusion]{basicmath} % disable protrusion for tt fonts
}{}
\makeatletter
\@ifundefined{KOMAClassName}{% if non-KOMA class
  \IfFileExists{parskip.sty}{%
    \usepackage{parskip}
  }{% else
    \setlength{\parindent}{0pt}
    \setlength{\parskip}{6pt plus 2pt minus 1pt}}
}{% if KOMA class
  \KOMAoptions{parskip=half}}
\makeatother
\usepackage{xcolor}
\IfFileExists{xurl.sty}{\usepackage{xurl}}{} % add URL line breaks if available
\IfFileExists{bookmark.sty}{\usepackage{bookmark}}{\usepackage{hyperref}}
\hypersetup{
  pdftitle={Selected Topics In Data Science},
  pdfauthor={Bruce Campbell},
  hidelinks,
  pdfcreator={LaTeX via pandoc}}
\urlstyle{same} % disable monospaced font for URLs
\usepackage{longtable,booktabs}
% Correct order of tables after \paragraph or \subparagraph
\usepackage{etoolbox}
\makeatletter
\patchcmd\longtable{\par}{\if@noskipsec\mbox{}\fi\par}{}{}
\makeatother
% Allow footnotes in longtable head/foot
\IfFileExists{footnotehyper.sty}{\usepackage{footnotehyper}}{\usepackage{footnote}}
\makesavenoteenv{longtable}
\usepackage{graphicx,grffile}
\makeatletter
\def\maxwidth{\ifdim\Gin@nat@width>\linewidth\linewidth\else\Gin@nat@width\fi}
\def\maxheight{\ifdim\Gin@nat@height>\textheight\textheight\else\Gin@nat@height\fi}
\makeatother
% Scale images if necessary, so that they will not overflow the page
% margins by default, and it is still possible to overwrite the defaults
% using explicit options in \includegraphics[width, height, ...]{}
\setkeys{Gin}{width=\maxwidth,height=\maxheight,keepaspectratio}
% Set default figure placement to htbp
\makeatletter
\def\fps@figure{htbp}
\makeatother
\setlength{\emergencystretch}{3em} % prevent overfull lines
\providecommand{\tightlist}{%
  \setlength{\itemsep}{0pt}\setlength{\parskip}{0pt}}
\setcounter{secnumdepth}{5}
\usepackage{booktabs}
\usepackage[]{natbib}
\bibliographystyle{apalike}

\title{Selected Topics In Data Science}
\author{Bruce Campbell}
\date{2020-12-29}

\begin{document}
\maketitle

{
\setcounter{tocdepth}{1}
\tableofcontents
}
\hypertarget{preface}{%
\chapter{Preface}\label{preface}}

This is the first installment on my promise to elucidate less popular topics in statistics and machine learning. I wrote this as a way to solidify my understanding of some of the topics that are treated here. Hopefully others will find value here.

\hypertarget{intro}{%
\chapter{Introduction}\label{intro}}

This is a living book. It's under development. We are using the \textbf{bookdown} package \citep{R-bookdown} in this book, which was built on top of R Markdown and \textbf{knitr} \citep{xie2015}.

\hypertarget{on-model-averaging}{%
\chapter{On Model Averaging}\label{on-model-averaging}}

Recall that we can break down model error into the bias an variance \(bias(\hat{Y})= E[\hat{Y}-E[Y]]\)

If we are averaging models \(i=1, \cdots ,k\) then

\(\operatorname{MSE}\left(\hat{Y}_{i}\right)=\left\{\operatorname{bias}\left(\hat{Y}_{i}\right)\right\}^{2}+\operatorname{var}\left(\hat{Y}_{i}\right)\)

\hypertarget{sensitivity-analysis-and-shapley-values}{%
\chapter{Sensitivity Analysis and Shapley Values}\label{sensitivity-analysis-and-shapley-values}}

Global sensitivity analysis measures the importance of input variables to a function. This is an important task in quantifying the uncertainty in which target variables can be predicted from their inputs. Sobol indices are a popular approach to this. It turns out that there's a relationship between Sobol indices and Shapley values. We explore this relationship here and demonstrate their effectiveness on some linear and non-linear models.

\hypertarget{relationship-between-sobol-indices-and-shapley-values}{%
\section{Relationship between Sobol indices and Shapley values}\label{relationship-between-sobol-indices-and-shapley-values}}

Shapley values are based on \(f(x)-E[f(x)]\) while Sobol indices decompose output variance into fractions contributed by the inputs. The Sobol index is a global measure of feature importance while Shapley values focus on local explanations although we could combine local Shapley values to achieve a global importance measure. Sobol indices are based on expectations and can be used for features not included in the model / function of interest. In this way we could query for important features correlated with those that the model does use.

\hypertarget{applications}{%
\chapter{Applications}\label{applications}}

Some \emph{significant} applications are demonstrated in this chapter.

\hypertarget{example-one}{%
\section{Example one}\label{example-one}}

\hypertarget{example-two}{%
\section{Example two}\label{example-two}}

\hypertarget{final-words}{%
\chapter{Final Words}\label{final-words}}

We have finished a nice book.

  \bibliography{book.bib,packages.bib}

\end{document}
